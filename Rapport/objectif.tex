\chapter{Objectif}

Le but de l'analyse est de répondre à la question "\textbf{Qui aime quoi?}". Le résultat sera une liste d'affirmations, par exemple, "\emph{Les hommes qui ont la trentaine aiment les films d'actions}".\\
\vspace{5mm}
\\Il est important de remarquer qu'une affirmation peut-être:
\begin{itemize}
  \item Plus ou moins vraie
  \item Plus ou moins détaillée
\end{itemize}
Evidemment seulement les affirmations qui sont nettement visibles seront gardées. Il n'est pas dit que des groupes d'utilisateurs existent. Au contraire le fait qu'il n'y en ait pas est aussi une information.
\vspace{5mm}
\\L'intuition devrait nous dire qu'on peut trouver des groupes plus ou moins précis. Le but est de trouver des groupes généralistes, peut-être dans l'optique de faire de la recommandation. Par exemple est-il plus intéressant de savoir que 90 pourcent des hommes du Midwest aiment les films d'actions que de savoir que 99 pourcent des avocats hommes du Midwest ayant la trentaine aiment les films d'action-aventure? Sachant que le deuxième groupe est plus restreint que le premier.